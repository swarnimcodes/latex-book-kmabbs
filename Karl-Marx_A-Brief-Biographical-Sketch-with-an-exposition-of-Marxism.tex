\documentclass[a4paper,12pt]{book}
\usepackage{}

\title{Karl Marx: A Brief Biographical Sketch with an Exposition of Marxism}
\author{Vladimir Ilyich Lenin}
\date{1914}
\pagestyle{headings}


\newcommand{\q}[1]{``#1''}


\begin{document}
\frontmatter
\maketitle
%\tableofcontents
\chapter*{Preface}
\addcontentsline{toc}{chapter}{\protect\numberline{}Preface}

This article on Karl Marx, which now appears in a separate printing, was written in 1913 (as far as I can remember) for the Granat Encyclopaedia. A fairly detailed bibliography of literature on Marx, mostly foreign, was appended to the article. This has been omitted in the present edition. The editor of the Encyclopaedia, for their part, have, for censorship reasons, deleted the end of the article on Marx, namely, the section dealing with his revolutionary tactics. Unfortunately, I am unable to reproduce that end, because the draft has remained among my papers somewhere in Krakow or in Switzerland. I only remember that in the concluding part of the article I quoted, among other things, the passage from Marx’s letter to Engels of April 16, 1856, in which he wrote: \q{The whole thing in Germany will depend on the possibility of backing the proletarian revolution by some second edition of the Peasant War. Then the affair will be splendid.} That is what our Mensheviks, who have now sunk to utter betrayal of socialism and to desertion to the bourgeoisie, have failed to understand since 1905.

\begin{flushright}
  \textbf{N. Lenin}
  
Moscow, May 14, 1918\footnote{Published in 1918 in the pamphlet: N. Lenin, Karl Marx, Priobi Publishers, Moscow. Published according to the manuscript}
\end{flushright}




\mainmatter

\textbf{\emph{Marx, Karl}}, was born on May 5, 1818 (New Style), in the city of Trier (Rhenish Prussia). His father was a lawyer, a Jew, who in 1824 adopted Protestantism. The family was well-to-do, cultured, but not revolutionary. After graduating from a \emph{Gymnasium}\footnote{Gymnasium [Germany] -- state-maintained secondary school that prepares pupils for higher academic education. This type of nine-year school originated in Strassburg in 1537. } in Trier, Marx entered the university, first at Bonn and later in Berlin, where he read law, majoring in history and philosophy. He concluded his university course in 1841, submitting a doctoral thesis on \emph{the philosophy of Epicurus}. At the time Marx was a Hegelian idealist in his views. In Berlin, he belonged to the circle of \q{Left Hegelians} (Bruno Bauer and others) who sought to draw atheistic and revolutionary conclusion from Hegel’s philosophy.

After graduating, Marx moved to Bonn, hoping to become a professor. However, the reactionary policy of the government, which deprived Ludwig Feuerbach of his chair in 1832, refused to allow him to return to the university in 1836, and in 1841 forbade young Professor Bruno Bauer to lecture at Bonn, made Marx abandon the idea of an academic career. Left Hegelian views were making rapid headway in Germany at the time. Feuerbach began to criticize theology, particularly after 1836, and turn to materialism, which in 1841 gained ascendancy in his philosophy (\emph{The Essence of Christianity}). The year 1843 saw the appearance of his \emph{Principles of the Philosophy of the Future}. \q{One must oneself have experienced the liberating effect} of these books, Engels subsequently wrote of these works of Feuerbach. \q{We [i.e., the Left Hegelians, including Marx] all became at once Feuerbachians.} At that time, some radical bourgeois in the Rhineland, who were in touch with the Left Hegelians, founded, in Cologne, an opposition paper called \emph{Rheinische Zeitung} (The first issue appeared on January 1, 1842). Marx and Bruno Bauer were invited to be the chief contributors, and in October 1842 Marx became editor-in-chief and moved from Bonn to Cologne. The newspaper’s revolutionary-democratic trend became more and more pronounced under Marx’s editorship, and the government first imposed double and triple censorship on the paper, and then on January 1 1843 decided to suppress it. Marx had to resign the editorship before that date, but his resignation did not save the paper, which suspended publication in March 1843. Of the major articles Marx contributed to \emph{Rheinische Zeitung}, Engels notes, in addition to those indicated below (see Bibliography\footnote{This “Bibliography” written by Lenin for the article is not included. -- Ed.}), an article on the condition of peasant winegrowers in the Moselle Valley\footnote{The reference is to the article “Justification of the Correspondent from the Mosel” by Karl Marx. -- Ed.}. Marx’s journalistic activities convinced him that he was insufficiently acquainted with political economy, and he zealously set out to study it.

In 1843, Marx married, at Kreuznach, a childhood friend he had become engaged to while still a student. His wife came of a reactionary family of the Prussian nobility, her elder brother being Prussia’s Minister of the Interior during a most reactionary period -- 1850--58. In the autumn of 1843, Marx went to Paris in order to publish a radical journal abroad, together with Arnold Ruge (1802--1880); Left Hegelian; in prison in 1825--30; a political exile following 1848, and a Bismarckian after 1866--70). Only one issue of this journal, \emph{Deutsch-Französische Jahrbücher}\footnote{The reference is to the \textit{Deutsch-Franz\"osische Jahrbucher} (German-French Annals), a magazine edited by Karl Marx and Arnold Ruge and published in German in Paris. Only the first issue, a double one, appeared, in February 1844. It included works by Karl Marx and Frederick Engels which marked the final transition of Marx and Engels to materialism and communism. Publication of the magazine was discontinued mainly as a result of basic differences of opinion between Marx and the bourgeois radical Ruge. -- Ed.}, appeared; publication was discontinued owing to the difficulty of secretly distributing it in Germany, and to disagreement with Ruge. Marx’s articles in this journal showed that he was already a revolutionary who advocated \q{merciless criticism of everything existing}, and in particular the \q{criticism by weapon}\footnote{These words are from Marx’s “Critique of the Hegelian Philosophy of Right: Introduction.” The relevant passage reads: “The weapon of criticism cannot, of course, replace criticism by weapon, material force must be overthrown by a material force; but theory, too, becomes a material force, as soon as it grips the masses.” -- \textit{Lenin}}, and appealed to the masses and to the proletariat.

 In September 1844, Frederick Engels came to Paris for a few days, and from that time on became Marx’s closest friend. They both took a most active part in the then seething life of the revolutionary groups in Paris (of particular importance at the time was Proudhon’s\footnote{\textit{Proudhonism} — An unscientific trend in petty-bourgeois socialism, hostile to Marxism, so called after its ideologist, the French anarchist \emph{Pierre Joseph Proudhon}. Proudhon criticized big capitalist property from the petty-bourgeois position and dreamed of perpetuating small private ownership. He proposed the foundation of “people’s” and “exchange” banks, with the aid of which the workers would be able to acquire the means of production, become handicraftsmen and ensure the just marketing of their produce. Proudhon did not understand the historic role of the proletariat and displayed a negative attitude to the class struggle, the proletarian revolution, and the dictatorship of the proletariat; as an anarchist, he denied the need for the state. Marx subjected Proudhonism to ruthless criticism in his work \emph{The Poverty of Philosophy}. - Ed.} doctrine), which Marx pulled to pieces in his \emph{Poverty of Philosophy, 1847}); waging a vigorous struggle against the various doctrines of petty-bourgeois socialism, they worked out the theory and tactics of revolutionary proletarian socialism, or communism Marxism). See Marx’s works of this period, 1844--48 in the Bibliography. At the insistent request of the Prussian government, Marx was banished from Paris in 1845, as a dangerous revolutionary. He went to Brussels. In the spring of 1847 Marx and Engels joined a secret propaganda society called the \emph{Communist League}\footnote{\emph{The Communist League} — The first international communist organization of the proletariat founded under the guidance of Marx and Engels in London early in June 1847. Marx and Engels helped to work out the programmatic and organizational principles of the League; they wrote its programme—the Manifesto of the Communist Party, published in February 1848. The Communist League was the predecessor of the International Working Men’s Association (The First International). It existed until November 1852, its prominent members later playing a leading role in the First International. -- Ed.}; they took a prominent part in the League’s Second Congress (London, November 1847), at whose request they drew up the celebrated \emph{Communist Manifesto}, which appeared in February 1848. With the clarity and brilliance of genius, this work outlines a new world-conception, consistent with materialism, which also embrace the realm of social life; dialectics, as the most comprehensive and profound doctrine of development; the theory of the class struggle and of the world-historic revolutionary role of the proletariat—the creator of a new, communist society.

On the outbreak of the Revolution of February 1848\footnote{The reference is to the bourgeois revolutions in Germany and Austria which began in March 1848. -- Ed.}, Marx was banished from Belgium. He returned to Paris, whence, after the March Revolution\footnote{ The reference is to the bourgeois revolution in France in February 1848. -- Ed.}, he went to Cologne, Germany, where \emph{Neue Rheinische Zeitung}\footnote{\emph{Die Neue Rheinische Zeitung} (New Rhenish Gazette)—Published in Cologne from June 1, 1848, to May 19, 1849. Marx and Engels directed the newspaper, Marx being its editor-in-chief. Lenin characterized Die Neue Rheinische Zeitung as “the finest and unsurpassed organ of the revolutionary proletariat”. Despite persecution and the obstacles placed in its way by the police, the newspaper staunchly defended the interests of revolutionary democracy, the interests of the proletariat. Because of Marx’s banishment from Prussia in May 1849 and the persecution of the other editors. Die Neue Rheinische Zeitung had to cease publication. -- Ed.} was published from June 1, 1848, to May 19, 1849, with Marx as editor-in-chief. The new theory was splendidly confirmed by the course of the revolutionary events of 1848-49, just as it has been subsequently confirmed by all proletarian and democratic movements in all countries of the world. The victorious counter-revolution first instigated court proceedings against Marx (he was acquitted on February 9, 1849), and then banished him from Germany (May 16, 1849). First Marx went to Paris, was again banished after the demonstration of June 13, 1849\footnote{The reference is to the mass demonstration in Paris organized by the Montagne, the party of the petty bourgeoisie, in protest against the infringement by the President and the majority in the Legislative Assembly of the constitutional orders established in the revolution of 1848. The demonstration was dispersed by the government. - Ed.}, and then went to London, where he lived until his death.

His life as a political exile was a very hard one, as the correspondence between Marx and Engels (published in 1913) clearly reveals. Poverty weighed heavily on Marx and his family; had it not been for Engels’ constant and selfless financial aid, Marx would not only have been unable to complete \emph{Capital} but would have inevitably have been crushed by want. Moreover, the prevailing doctrines and trends of petty-bourgeois socialism, and of non-proletarian socialism in general, forced Marx to wage a continuous and merciless struggle and sometime to repel the most savage and monstrous personal attacks (\emph{Herr Vogt}\footnote{The reference is to Marx’s pamphlet \emph{Herr Vogt}, which was written in reply to the slanderous pamphlet by Vogt, a Bonapartist agent provocateur, \emph{My Process Against “Allgemeine Zeitung”}. -- Ed.}). Marx, who stood aloof from circles of political exiles, developed his materialist theory in a number of historical works (see Bibliography), devoting himself mainly to a study of political economy. Marx revolutionized science (see “The Marxist Doctrine”, below) in his Contribution to the Critique of Political Economy (1859) and Capital (Vol. I, 1867).

The revival of the democratic movements in the late fifties and in the sixties recalled Marx to practical activity. In 1864 (September 28) the International Working Men’s Association—the celebrated First International—was founded in London. Marx was the heart and soul of this organization, and author of its first Address\footnote{The First International Workingmen’s Association was the first international tendency that grouped together all the worlds’ workers parties in one unified international party. -- Ed.} and of a host of resolutions, declaration and manifestoes. In uniting the labor movement of various forms of non-proletarian, pre-Marxist socialism (Mazzini, Proudhon, Bakunin, liberal trade-unionism in Britain, Lassallean vacillations to the right in Germany, etc.), and in combating the theories of all these sects and schools, Marx hammered out a uniform tactic for the proletarian struggle of the working in the various countries. Following the downfall of the Paris Commune (1871)—of which gave such a profound, clear-cut, brilliant effective and revolutionary analysis (The Civil War In France, 1871)—and the Bakunin\footnote{\emph{Bakuninism} -- A trend called after its leader Mikhail Bakunin, an ideologist of anarchism and enemy of Marxism and scientific socialism. -- Ed. }-caused cleavage in the International, the latter organization could no longer exist in Europe. After the Hague Congress of the International (1872), Marx had the General Council of the International had played its historical part, and now made way for a period of a far greater development of the labor movement in all countries in the world, a period in which the movement grew in scope, and mass socialist working-class parties in individual national states were formed.

Marx’s health was undermined by his strenuous work in the International and his still more strenuous theoretical occupations. He continued work on the refashioning of political economy and on the completion of \emph{Capital}, for which he collected a mass of new material and studied a number of languages (Russian, for instance). However, ill-health prevented him from completing \emph{Capital}.

His wife died on December 2, 1881, and on March 14, 1883, Marx passed away peacefully in his armchair. He lies buried next to his wife at Highgate Cemetery in London. Of Marx’s children some died in childhood in London, when the family were living in destitute circumstances. Three daughters married English and French socialists; Eleanor Aveling, Laura Lafargue and Jenny Longuet. The latters’ son is a member of the French Socialist Party. 


\chapter{The Marxist Doctrine}
Marxism is the system of Marx’s views and teachings. Marx was the genius who continued and consummated the three main ideological currents of the 19th century, as represented by the three most advanced countries of mankind: classical German philosophy, classical English political economy, and French socialism combined with French revolutionary doctrines in general. Acknowledged even by his opponents, the remarkable consistency and integrity of Marx’s views, whose totality constitutes modern materialism and modern scientific socialism, as the theory and programme of the working-class movement in all the civilized countries of the world, make it incumbent on us to present a brief outline of his world-conception in general, prior to giving an exposition of the principal content of Marxism, namely, Marx’s economic doctrine.

\section*{Philosophical Materialism}

Beginning with the years 1844–45, when his views took shape, Marx was a materialist and especially a follower of Ludwig Feuerbach, whose weak point he subsequently saw only in his materialism being insufficiently consistent and comprehensive. To Marx, Feuerbach’s historic and “epoch-making” significance lay in his having resolutely broken with Hegel’s idealism and in his proclamation of materialism, which already “in the 18th century, particularly French materialism, was not only a struggle against the existing political institutions and against... religion and theology, but also... against all metaphysics” (in the sense of “drunken speculation” as distinct from “sober philosophy”). (\emph{The Holy Family, in Literarischer Nachlass}\footnote{See Marx and Engels, The Holy Family (Chapter Eight) -- \emph{Lenin}}) “To Hegel... ,” wrote Marx, “the process of thinking, which, under the name of ‘the Idea’, he even transforms into an independent subject, is the demiurgos (the creator, the maker) of the real world.... With me, on the contrary, the ideal is nothing else than the material world reflected by the human mind, and translated into forms of thought.” (Capital, Vol. I, Afterward to the Second Edition.) In full conformity with this materialist philosophy of Marx’s, and expounding it, Frederick Engels wrote in Anti-Duhring (read by Marx in the manuscript): “The real unity of the world consists in its materiality, and this is proved... by a long and wearisome development of philosophy and natural science....” “Motion is the mode of existence of matter. Never anywhere has there been matter without motion, or motion without matter, nor can there be.... Bit if the... question is raised: what thought and consciousness really are, and where they come from; it becomes apparent that they are products of the human brain and that man himself is a product of Nature, which has developed in and along with its environment; hence it is self-evident that the products of the human brain, being in the last analysis also products of Nature, do not contradict the rest of Nature’s interconnections but are in correspondence with them....

“Hegel was an idealist, that is to say, the thoughts within his mind were to him not the more or less abstract images [\emph{Abbilder}, reflections; Engels sometimes speaks of “imprints”] of real things and processes, but on the contrary, things and their development were to him only the images, made real, of the “Idea” existing somewhere or other before the world existed.”

In his \emph{Ludwig Feuerbach} -- which expounded his own and Marx’s views on Feuerbach’s philosophy, and was sent to the printers after he had re-read an old manuscript Marx and himself had written in 1844-45 on Hegel, Feuerbach and the materialist conception of history—Engels wrote:

\begin{center}
\footnotesize{\q{The great basic question of all philosophy, especially of more recent philosophy, is the relation of thinking and being... spirit to Nature... which is primary, spirit or Nature.... The answers which the philosophers gave to this question split them into two great camps. Those who asserted the primary of spirit to Nature and, therefore, in the last instance, assumed world creation in some form or other... comprised the camp of idealism. The others, who regarded Nature as primary, belonged to the various schools of materialism.}}
\end{center}

 Any other use of the concepts of (philosophical) idealism and materialism leads only to confusion. Marx decidedly rejected, not only idealism, which is always linked in one way or another with religion, but also the views—especially widespread in our day—of Hume and Kant, agnosticism, criticism, and positivism\footnote{\emph{Agnoticism} -- An idealist philosophical theory asserting that the world in unknowable, that the human mind is limited and cannot know anything beyond the realms of sensations. Agnosticism has various forms: some agnostics recognize the objective existence of the material world but deny the possibility of knowing it, others deny the existence of the material world on the plea that man cannot know whether anything exists beyond his sensations.

\emph{Criticism} -- Kant gave this name to his idealist philosophy, considering the criticism of man’s cognitive ability to be the purpose of that philosophy. Kant’s criticism led him to the conviction that human reason cannot know the nature of things.

\emph{Positivism} -- A widespread trend in bourgeois philosophy and sociology, founded by Comte (1798-1857), a French philosopher and sociologist. The positivists deny the possibility of knowing inner regularities and relations and deny the significance of philosophy as a method of knowing and changing the objective world. They reduce philosophy to a summary of the data provided by the various branches of science and to a superficial description of the results of direct observation—i.e., to “positive” facts. Positivism considers itself to be “above” both materialism and idealism but it is actually nothing more than a variety of subjective idealism.—Ed.} in their various forms; he considered that philosophy a “reactionary” concession to idealism, and at best a “shame-faced way of surreptitiously accepting materialism, while denying it before the world.”\footnote{Frederick Engels: Ludwig Feuerbach and the End of Classical German Philosophy -- \emph{Lenin}}

On this question, see, besides the works by Engels and Marx mentioned above, a letter Marx wrote to Engels on December 12, 1868, in which, referring to an utterance by the naturalist Thomas Huxley, which was “more materialistic” than usual, and to his recognition that “as long as we actually observe and think, we cannot possibly get away from materialism”, Marx reproached Huxley for leaving a “loop hole” for agnosticism, for Humism.

It is particularly important to note Marx’s view on the relation between freedom and necessity: “Freedom is the appreciation of necessity. ‘Necessity is blind only insofar as it is not understood.’” (Engels in \emph{Anti-Duhring}) This means recognition of the rule of objective laws in Nature and of the dialectical transformation of necessity into freedom (in the same manner as the transformation of the uncognized but cognizable “thing-in-itself” into the “thing-for-us”, of the “essence of things” into “phenomena”). Marx and Engels considered that the “old” materialism, including that of Feuerbach (and still more the “vulgar” materialism of Buchner, Vogt and Moleschott), contained the following major shortcomings:

(1) this materialism was “predominantly mechanical,” failing to take account of the latest developments in chemistry and biology (today it would be (2) necessary to add: and in the electrical theory of matter); the old materialism was non-historical and non-dialectical (3) (metaphysical, in the meaning of anti-dialectical), and did not adhere consistently and comprehensively to the standpoint of development; it regarded the “human essence” in the abstract, not as the “complex of all” (concretely and historically determined) “social relations”, and therefore merely “interpreted” the world, whereas it was a question of “changing” it, i.e., it did not understand the importance of “revolutionary practical activity”.

\section*{Dialectics}
As the most comprehensive and profound doctrine of development, and the richest in content, Hegelian dialectics was considered by Marx and Engels the greatest achievement of classical German philosophy. They thought that any other formulation of the principle of development, of evolution, was one-sided and poor in content, and could only distort and mutilate the actual course of development (which often proceeds by leaps, and via catastrophes and revolutions) in Nature and in society.

“Marx and I were pretty well the only people to rescue conscious dialectics [from the destruction of idealism, including Hegelianism] and apply it in the materialist conception of Nature.... Nature is the proof of dialectics, and it must be said for modern natural science that it has furnished extremely rich [this was written before the discovery of radium, electrons, the transmutation of elements, etc.!] and daily increasing materials for this test, and has thus proved that in the last analysis Nature’s process is dialectical and not metaphysical.

“ The great basic thought,” Engels writes, “that the world is not to be comprehended as a complex of ready-made things, but as a complex of processes, in which the things apparently stable no less than their mind images in our heads, the concepts, go through an uninterrupted change of coming into being and passing away... this great fundamental thought has, especially since the time of Hegel, so thoroughly permeated ordinary consciousness that in this generality it is now scarcely ever contradicted. But to acknowledge this fundamental thought in words and to apply it in reality in detail to each domain of investigation are two different things.... For dialectical philosophy nothing is final, absolute, sacred. It reveals the transitory character of everything and in everything; nothing can endure before it except the uninterrupted process of becoming and of passing away, of endless ascendancy from the lower to the higher. And dialectical philosophy itself is nothing more than the mere reflection of this process in the thinking brain.” Thus, according to Marx, dialectics is “the science of the general laws of motion, both of the external world and of human thought.”\footnote{Frederick Engels: Ludwig Feuerbach and the End of Classical German Philosophy -- \emph{Lenin}}

This revolutionary aspect of Hegel’s philosophy was adopted and developed by Marx. Dialectical materialism “does not need any philosophy standing above the other sciences.” From previous philosophy there remains “the science of thought and its laws—formal logic and dialectics.” Dialectics, as understood by Marx, and also in conformity with Hegel, includes what is now called the theory of knowledge, or epistemology, studying and generalizing the original and development of knowledge, the transition from \emph{non}-knowledge to knowledge.

In our times, the idea of development, of evolution, has almost completely penetrated social consciousness, only in other ways, and not through Hegelian philosophy. Still, this idea, as formulated by Marx and Engels on the basis of Hegels’ philosophy, is far more comprehensive and far richer in content than the current idea of evolution is. A development that repeats, as it were, stages that have already been passed, but repeats them in a different way, on a higher basis (“the negation of the negation”), a development, so to speak, that proceeds in spirals, not in a straight line; a development by leaps, catastrophes, and revolutions; “breaks in continuity”; the transformation of quantity into quality; inner impulses towards development, imparted by the contradiction and conflict of the various forces and tendencies acting on a given body, or within a given phenomenon, or within a given society; the interdependence and the closest and indissoluble connection between \emph{all} aspects of any phenomenon (history constantly revealing ever new aspects), a connection that provides a uniform, and universal process of motion, one that follows definite laws—these are some of the features of dialectics as a doctrine of development that is richer than the conventional one. (Cf. Marx’s letter to Engels of January 8, 1868, in which he ridicules Stein’s “wooden trichotomies,” which it would be absurd to confuse with materialist dialectics.)

\section*{The Materialist Conception of History}

A realization of the inconsistency, incompleteness, and onesidedness of the old materialism convinced Marx of the necessity of “bringing the science of society\ldots{} into harmony with the materialist foundation, and of reconstructing it thereupon.”\footnote{Frederick Engels: Ludwig Feuerbach and the End of Classical German Philosophy -- \emph{Lenin}} Since materialism in general explains consciousness as the outcome of being, and not conversely, then materialism as applied to the social life of mankind has to explain social consciousness as the outcome of social being. “Technology,” Marx writes (\emph{Capital, Vol. I}), “discloses man’s mode of dealing with Nature, the immediate process of production by which he sustains his life, and thereby also lays bare the mode of formation of his social relations, and of the mental conceptions that flow from them.”\footnote{See Karl Marx, Capital. Volume I -- \emph{Lenin}} In the preface to his \emph{Contribution to the Critique of Political Economy}, Marx gives an integral formulation of the fundamental principles of materialism as applied to human society and its history, in the following words:

\begin{center}
\footnotesize{ “In the social production of their life, men enter into definite relations that are indispensable and independent of their will, relations of production which correspond to a definite stage of development of their material productive forces.

“The sum total of these relations of production constitutes the economic structure of society, the real foundation, on which rises a legal and political superstructure and to which correspond definite forms of social consciousness. The mode of production of material life conditions the social, political and intellectual life process in general. It is not the consciousness of men that determines their being, but, on the contrary, their social being that determines their consciousness. At a certain stage of their development, the material productive forces of society come in conflict with the existing relations of production, or—what is but a legal expression for the same thing—with the property relations within which they have been at work hitherto. From forms of development of the productive forces these relations turn into their fetters. Then begins an epoch of social revolution. With the change of the economic foundation the entire immense superstructure is more or less rapidly transformed. In considering such transformations a distinction should always be made between the material transformation of the economic conditions of production, which can be determined with the precision of natural science, and the legal, political, religious, aesthetic or philosophic—in short, ideological forms in which men become conscious of this conflict and fight it out.

“Just as our opinion of an individual is not based on what he thinks of himself, so we cannot judge of such a period of transformation by its own consciousness; on the contrary, this consciousness must be explained rather from the contradictions of material life, from the existing conflict between the social productive forces and the relations of production.... In broad outlines, Asiatic, ancient, feudal, and modern bourgeois modes of production can be designated as progressive epochs in the economic formation of society.”\footnote{Karl Marx, Contribution to the Critique of Political Economy (1859) \emph{Lenin}} [Cf. Marx’s brief formulation in a letter to Engels dated July 7, 1866: “Our theory that the organization of labor is determined by the means of production.”]}
\end{center}

The discovery of the materialist conception of history, or more correctly, the consistent continuation and extension of materialism into the domain of social phenomena, removed the two chief shortcomings in earlier historical theories. In the first place, the latter at best examined only the ideological motives in the historical activities of human beings, without investigating the origins of those motives, or ascertaining the objective laws governing the development of the system of social relations, or seeing the roots of these relations in the degree of development reached by material production; in the second place, the earlier theories did not embrace the activities of the masses of the population, whereas historical materialism made it possible for the first time to study with scientific accuracy the social conditions of the life of the \emph{masses}, and the changes in those conditions. \emph{At best}, pre-Marxist “sociology” and historiography brought forth an accumulation of raw facts, collected at random, and a description of individual aspects of the historical process. By examining the \emph{totality} of opposing tendencies, by reducing them to precisely definable conditions of life and production of the various classes of individual aspects of the historical process. By examining the choice of a particular “dominant” idea or in its interpretation, and by revealing that, without exception, all ideas and all the various tendencies stem from the condition of the material forces of production, Marxism indicated the way to an all-embracing and comprehensive study of the process of the rise, development, and decline of socio-economic systems. People make their own history but what determines the motives of people, of the mass of people—i.e., what is the sum total of all these clashes in the mass of human societies? What are the objective conditions of production of material life that form the basis of all man’s historical activity? What is the law of development of these conditions? To all these Marx drew attention and indicated the way to a scientific study of history as a single process which, with all its immense variety and contradictoriness, is governed by definite laws.

\section*{The Class Struggle}

It is common knowledge that, in any given society, the striving of some of its members conflict with the strivings of others, that social life is full of contradictions, and that history reveals a struggle between nations and societies, as well as within nations and societies, and, besides, an alternation of periods of revolution and reaction, peace and war, stagnation and rapid progress or decline. Marxism has provided the guidance —i.e., the theory of the class struggle—for the discovery of the laws governing this seeming maze and chaos. It is only a study of the sum of the strivings of all the members of a given society or group of societies that can lead to a scientific definition of the result of those strivings. Now the conflicting strivings stem from the difference in the position and mode of life of the classes into which each society is divided.

“The history of all hitherto existing society is the history of class struggles,” Marx wrote in the \emph{Communist Manifesto} (with the exception of the history of the primitive community, Engels added subsequently). “Freeman and slave, patrician and plebeian, lord and serf, guild-master and journeyman, in a word, oppressor and oppressed, stood in constant opposition to one another, carried on an uninterrupted, now hidden, now open fight, a fight that each time ended, either in a revolutionary reconstruction of society at large, or in the common ruin of the contending classes.... The modern bourgeois society that has sprouted from the ruins of feudal society has not done away with class antagonisms. It has but established new classes, new conditions of oppression, new forms of struggle in place of the old ones. Our epoch, the epoch of the bourgeoisie, possesses, however, this distinctive feature: it has simplified class antagonisms. Society as a whole is more and more splitting up into two great hostile camps, into two great classes directly facing each other: Bourgeoisie and Proletariat.”

Ever since the Great French Revolution, European history has, in a number of countries, tellingly revealed what actually lies at the bottom of events—the struggle of classes. The Restoration period in France\footnote{\emph{The Restoration} -- The period in France between 1814 and 1830 when power was in the hands of the Bourbons, restored to the throne after their overthrow by the French bourgeois revolution in 1792. -- Ed.} already produced a number of historians (Thierry, Guizot, Mignet, and Thiers) who, in summing up what was taking place, were obliged to admit that the class struggle was taking place, were obliged to admit that the class struggle was the key to all French history. The modern period—that of complete victory of the bourgeoisie, representative institutions, extensive (if not universal) suffrage, a cheap daily press that is widely circulated among the masses, etc., a period of powerful and ever-expanding unions of workers and unions of employers, etc.—has shown even more strikingly (though sometimes in a very one-sided, “peaceful”, and “constitutional” form) the class struggle as the mainspring of events. The following passage from Marx’s \emph{Communist Manifesto} will show us what Marx demanded of social science as regards an objective analysis of the position of each class in modern society, with reference to an analysis of each class’s conditions of development:

\begin{center}
  \footnotesize{“Of all the classes that stand face to face with the bourgeoisie today, the proletariat alone is a really revolutionary class. The other classes decay and finally disappear in the face of Modern Industry; the proletariat is its special and essential product. The lower middle class, the small manufacturer, the shopkeeper, the artisan, the peasant, all these fight against the bourgeoisie, to save from extinction their existence as fractions of the middle class. They are therefore not revolutionary, but conservative. Nay more, they are reactionary, for they try to roll back the wheel of history. If by chance they are revolutionary, they are so only in view of their impending transfer into the proletariat; they thus defend not their present, but their future interests; they desert their own standpoint to place themselves at that of the proletariat.”}
\end{center}

 In a number of historical works (see Bibliography), Marx gave brilliant and profound examples of materialist historiography, of an analysis of the position of each individual class, and sometimes of various groups or strata within a class, showing plainly why and how “every class struggle is a political struggle.”\footnote{See Marx and Engels, Selected Works Vol. 1, Moscow, 1973, pp. 108-09, 117-18, 116. —Ed.} The above-quoted passage is an illustration of what a complex network of social relations and transitional stages from one class to another, from the past to the future, was analyzed by Marx so as to determine the resultant of historical development.

Marx’s economic doctrine is the most profound, comprehensive and detailed confirmation and application of his theory.


\chapter{Marx's Economic Doctrine}

\chapter{Socialism}

\chapter{Tactics of the Class Struggle if the Proleteriat}

\appendix


\end{document}
